











%\subsection{Bespoke non-bonded forcefield parameters}\label{sec:ddec}
%%
%The Tkatchenko-Scheffler (TS) method is an approach to post process the ground state electronic density --- produced by the DFT calculations as outlined in (\ref{sec:linear_scaling}) --- to derive the non-bonded forcefield parameters from (Eq.~\ref{eq:forcefield}) for use in MD simulations.\cite{tkatchenko2009accurate} It utilises the charge density dependence of both atomic charges as well as Lennard-Jones parameters, therefore accounting for the impact of the local chemical environment on the van der Waals contributions and point charges of atoms. The TS requires an atoms-in-molecule method, which uses the ground state DFT electronic density of the molecular system ($\rho_{\rm{mol}}(\mathbf{r})$) to divide into into uniform overlapping atomic densities ($\rho_{A}(\mathbf{r})$) through a weighting factor $w_{A}(\mathbf{r})$ (Eq.~\ref{eq:TS}).\cite{tkatchenko2009accurate,manz2012improved}
%%
%\begin{equation} \label{eq:TS}
%\rho_{A}(\mathbf{r})=w_{A}(\mathbf{r})\rho_{\mathrm{mol}}(\mathbf{r})
%\end{equation}
%%
%The weighting factor is calculated by computing the share of each isolated atom ($\rho_{A}^0(\mathbf{r})$) of the ($N$) atoms that make up a promolecule ($\rho_{\mathrm{mol}}^0(\mathbf{r})$):
%\begin{equation}
%    \rho_{\mathrm{mol}}^0(\mathbf{r}) = \sum_{A=1}^N \rho_{A}^0(\mathbf{r})
%\end{equation}
%where the weighting factors are differently formulated according to the atoms-in-molecule scheme, in the iterative Hirshfeld (IH) scheme,\cite{bultinck2007uniqueness} it is given by:
%\begin{equation}
%    w_{A}^{\mathrm{IH}}(\mathbf{r}) = \frac{\rho_{A}^0(\mathbf{r})}{\rho_{\mathrm{mol}}^0(\mathbf{r})}
%\end{equation}
%The self-consistency of the IH scheme therefore requires that the share of the isolated atomic densities ($\rho_{A}^0(\mathbf{r})$) from the promolecular density ($\rho_{\mathrm{mol}}^0(\mathbf{r})$) is equivalent to the share of the overlapping atomic density ($\rho_{A}(\mathbf{r})$) from the total molecular density ($\rho_{\mathrm{mol}}(\mathbf{r})$). The atomic electron population ($N_A$) are derived from the atomic densities ($\rho_{A}(\mathbf{r})$) by:
%%
%\begin{equation}
%    N_A = \int \rho_A (\mathbf{r}) \mathrm{d}\mathbf{r}
%\end{equation}
%%
%In the IH scheme, the isolated atomic density is composed of the weighted average of the atomic densities with the next lowest integer ($\mathrm{lint}(N_A)$) and next highest integer ($\mathrm{uint}(N_A)$) occupancies. %
%\begin{equation}
%\begin{aligned}
%\rho_{A}^{0, N_{A}}(\mathbf{r})=& \,\,\rho_{A}^{0, \operatorname{lint}\left(N_{A}\right)}(\mathbf{r})\left[\operatorname{uint}\left(N_{A}\right)-N_{A}\right] \\
%&+\rho_{A}^{0, \operatorname{uint}\left(N_{A}\right)}(\mathbf{r})\left[N_{A}-\operatorname{lint}\left(N_{A}\right)\right]
%\end{aligned}
%\end{equation}
%The iterated stockholder atoms (ISA) scheme \cite{lillestolen2008redefining} the weighting factor is instead defined with respect to the spherical average around atom ($A$) by (Eq.~\ref{eq:ISA_weights}). Unlike the IH scheme, the self-consistency in ISA requires that every value of the radius of a sphere around each nucleus ($A$), the average electron density on the surface of this sphere is the same in the promolecular atom ($\langle \rho_{A}^{0, \mathrm{ISA}}(d)\rangle$) and in the atom in the molecule ($\langle \rho_{A}^{\mathrm{ISA}}(d)\rangle$).\cite{bultinck2009comparison}
%%
%\begin{equation} \label{eq:ISA_weights}
%w_{A}^{\mathrm{ISA}}(\mathbf{r})=\frac{\left\langle\rho_{A}^{0, \mathrm{ISA}}\left(\left|\mathbf{r}-\mathbf{R}_{A}\right|\right)\right\rangle}{\rho_{\mathrm{mol}}^{0, \mathrm{ISA}}(\mathbf{r})}
%\end{equation}
%where ($\langle\rho_{A}^{0, \mathrm{ISA}}\left(\left|\mathbf{r}-\mathbf{R}_{A}\right|\right)\rangle$) is the average density over the surface of a sphere with radius ($d$) for an isolated atom ($A$). The promolecular density in (Eq.~\ref{eq:ISA_weights}) is similarly defined as the sum of spherically averaged spheres of radii ($|\mathbf{r} - \mathbf{R}_B|$) around every atom ($B$):
%%
%\begin{equation}
%\rho_{\mathrm{mol}}^{0, \mathrm{ISA}}(\mathbf{r})=\sum_{B=1}^{N}\left\langle\rho_{B}^{0, \mathrm{ISA}}\left(\left|\mathbf{r}-\mathbf{R}_{B}\right|\right)\right\rangle
%\end{equation}
%The density derived electrostatic and chemical electron density partitioning (DDEC) method is one of the atoms-in-molecule electronic density partitioning approaches. DDEC uses a mixture of both IH and ISA methods and iteratively optimises for a converged solution to the weighing factor to resemble the spherical average of the atomic densities ($\rho_{A}(\mathbf{r})$) and the density of an isolated reference atom ($\rho_{A}^0(\mathbf{r})$). The atomic partial charges ($q_A$) for MD are derived from DDEC by (Eq.~\ref{eq:ddec_charge}), where ($Z_A$) is the nuclear charge of atom $A$. 
%%
%\begin{equation} \label{eq:ddec_charge}
%    q_A = Z_A - N_A = Z_A - \int \rho_A(\mathbf{r}) \mathrm{d}^3 \mathbf{r}
%\end{equation}
%%
%Whereas the dispersion ($B_{ij}$) and repulsion ($A_{ij}$) coefficients are derived from the partitioned electronic density by (Eq.~\ref{eq:qube_bij}) and (Eq.~\ref{eq:qube_aij}).\cite{horton2019qubekit}
%%Whereas the Lennard-Jones parameters from (Eq.~\ref{eq:forcefield}) are derived from the partitioned electronic density by (Eq.~\ref{eq:qube_ts}).\cite{horton2019qubekit}
%%
%\begin{subequations} 
%\begin{align}
%\begin{split}\label{eq:qube_bij}
%B_{i}=\left(\frac{V_{A}^{\mathrm{DDEC}}}{V_{A}^{\mathrm{free}}}\right)^{2} B_{i}^{\mathrm{free}}
%\end{split}\\
%\begin{split}\label{eq:qube_aij}
%A_{i}=\frac{1}{2} B_{i}\left(2 R_{A}^{\mathrm{DDEC}}\right)^{6}
%\end{split}
%\end{align}
%\end{subequations}
%%
%The atomic volume ($V_{A}^{\mathrm{DDEC}}$) is calculated from the electronic density:
%%
%\begin{equation} \label{eq:qube_volume}
%V_{A}^{\mathrm{DDEC}}=\int r^{3} \rho_{A}(\mathbf{r}) \mathrm{d}^{3} \mathbf{r}
%\end{equation}
%%
%Whereas the other terms are derived from alternative methods. Namely, the free dispersion coefficients ($B_{i}^{\mathrm{free}}$) are computed using time-dependent DFT calculations of free atoms in vacuum,\cite{chu2004linear} the reference volume ($V_{A}^{\mathrm{free}}$) is calculated using more accurate \textit{ab initio} approaches with explicit treatment of electron correlation effects and and the DDEC effective radius of each atom rescales the experimentally derived reference free atom radius ($R_{A}^{\mathrm{free}}$) using (Eq.~\ref{eq:qube_radius}).\cite{horton2019qubekit} 
%%
%\begin{equation} \label{eq:qube_radius}
%R_{A}^{\mathrm{DDEC}}=\left(\frac{V_{A}^{\mathrm{AIM}}}{V_{A}^{\mathrm{free}}}\right)^{1 / 3} R_{A}^{\mathrm{free}}
%\end{equation}
%%
%The dispersion and repulsion coefficients are related to the Lennard-Jones ($\epsilon$) and ($\sigma$) parameters (Eq.~\ref{eq:forcefield}) via $A_{ij} = 4 \epsilon_{ij}\,\sigma_{ij}^{12}$ and $B_{ij} = 4 \epsilon_{ij}\,\sigma_{ij}^{6}$. The resulting parameters are an accurate representation of the chemistry of the molecular system under investigation, since they are exclusively derived from \textit{ab initio} quantum mechanical calculations. As such, computing the non-bonded forcefield parameters using DDEC is a basis for bespoke forcefield parametrisation to bypass the use for transferable forcefields. This approach is implemented for small molecules by software such as QUBEKit.\cite{horton2019qubekit} 









%%%%%%%%%%%%%%%%%%%%%%%%%%%%%%%%%%%%%%


%\begin{subequations} \label{eq:ewald_sep}
%\begin{align}
%\begin{split}
%V_{tot} = V_{sr} + V_{lr} + V_{0} 
%\end{split}\\
%\begin{split}
%V_{sr} = \frac{f}{2} \sum_{i,j}^{N} \sum_{n_x}\sum_{n_y} \sum_{n_{z}*} q_i q_j \frac{\mbox{erfc}(\beta {r}_{ij,{\bf n}} )}{{r}_{ij,{\bf n}}}
%\end{split}\\
%\begin{split}
%V_{lr} = \frac{f}{2 \pi V} \sum_{i,j}^{N} q_i q_j \sum_{m_x}\sum_{m_y} \sum_{m_{z}*} \frac{\exp{\left( -(\pi {\bf m}/\beta)^2 + 2 \pi i {\bf m} \cdot ({\bf r}_i - {\bf r}_j)\right)}}{{\bf m}^2}  
%\end{split}\\
%\begin{split}
%V_{0} = -\frac{f \beta}{\sqrt{\pi}}\sum_{i}^{N} q_i^2 
%\end{split}
%\end{align}
%\end{subequations}
%
%    \begin{equation} \label{forcefield}
%    \begin{split}
%    U(\mathbf{r}) &= \sum_{bonds} K_b(r_b - r_0)^2 + \sum_{angles} K_{\theta}(\theta_a - \theta_0)^2 + \sum_{dihedrals} K_{\chi} (1+\cos{n_{\chi d} - \sigma}) \\
%    &+  \sum_{impropers} K_{\eta} (\phi_{\eta} - \theta_{0})^2 + \sum_{nonbonded} \left(\left[ \frac{C_{ij}^{(12)}}{r_{ij}^{12}}- \frac{C_{ij}^{(6)}}{{r_{ij}^{6}}} \right] \right) + \sum_{i < j}  \frac{q_{i}q_{j}}{r_{ij}}
%    \end{split}
%    \end{equation}
%\edit{This is a terrible paragraph, stick to the brief and discuss anything to do with the work in the motivation...} . They are parameterised by fitting to empirical properties of liquid or quantum binding energy data, a consequence of such an approximation means that propane and valine share the same parameters for all environments. This is an oversight that can result in deficiencies with experimental observables and what limits MD to a verification tool --- secondary to experiment, however an ideal description of a forcefield could promote it as the leading source of findings to guide experiment.  There are limitations that cannot be remedied, including the description of the electronic states of molecules, charge transfer effects and bond breaking/formation. The solution to bad classical parametrisation of molecules, particularly those containing functional groups, small ligands and transition metals is large scale DFT calculations. Linear-scaling DFT, as covered in (Chapter~\ref{chapter:onetep}), is capable of describing the electronic structure of systems of thousands of atoms. However, nothing is perfect; DFT itself is inaccurate for a smaller subset of exotic materials such as those studied in (Chapter~\ref{Hemocyanin}) where electronic correlation effects play an important role in the biological function and the phenomenological observables of the system. \\

% ---------------------------
%           QM FFs
% ---------------------------

\section{Quantum mechanical forcefield design}
This chapter will explain the motivation for using quantum chemical forcefields for use in molecular dynamics simulations. This will start by looking at forcefield development for different forcefields and the motivation for their individual approaches. We will consider research performed where out of the box forcefields were considered where they may have incorrectly described the physics of the system and used for the non-intended purpose. We will then provide a comparison between quantum derived forcefield parameters and their variance across a given system and study whether the approximations employed by normal forcefields may result in significant global scale differences in measurable properties. 

The Atoms in Molecule (AIM) approach using Density Functional Theory (DFT) is ………….. ? 

Density Derived Electronic Density partitioning (DDEC) uses electronic densities derived from DFT calculations to partition charge within atomic radii to give a partial charge to each atom in the system as required for molecular dynamics simulations. The Tschafchenko-something method uses the DDEC charges to in turn calculate the Lennard-Jones (LJ) parameters or the non-bonded potentials in the system (equation XX). 

The Tschafchenko method works by ………………………

The implementations of DFT typically scales cubically with the number of electrons in the system and given the size of the nanomaterials necessary to physically span the interface with a serum protein to study its denaturing and/or the thickness of a lipid membrane to study its penetration makes DFT-derived forcefields challenging, computationally costly and often out of reach. For materials with a large band-gap, the ONETEP DFT package can scale linearly with the number of electrons in the system. This was not utilised for the nanomaterial in question in this work, as the structure resulted in a vanishing band-gap and required full cubic scaling treatment. This does however prove useful for if and when the remainder of the biological system is treated with DFT. 

For our application, we cowrote a Python package to draw the coordinates or a graphitic molecule (graphene, graphene oxide, double-clickable graphene oxide) and determine the bonding network as well as the correct assignment of bonding parameters from the OPLS forcefield. This was further developed to convert the format of the forcefield to be readable in the GROMACS molecular dynamics engine software, which is a highly parallel code for use in both CPU and GPU architectures — this is important as the systems we are dealing with will reach up to a million atoms and contain systems with high degrees of freedom that require hundreds of nanoseconds of simulation time to converge to their global minimum. 

Out of the box forcefields are notoriously successful for their intended use — there exist a range of forcefields that are designed for use in any of DNA structures (AMBER?), proteins (CHARMM?), flexible proteins (another AMBER?) — these come in one of three forms — some are parametrised by fitting to 1) experimental data, 2) ab-initio calculations or 3) both. This does however mean that for most of these forcefields, any specific system will suffer the detriment of approximation to the fitting procedure. This can be especially costly if the system of interest is 1) a highly flexible ligand or drug molecule, as both charges and LJ parameters will be sensitive to such changes in geometry (discussed further in Background section — maybe relate this to Parsley and OpenFF project) and will require a scan of the potential energy surface to sufficiently sample the phase space to locate the global minimum. Such a scan often takes the form of an iterative procedure that rotates the flexible dihedral bonds for the ligand of interest and running a quantum chemical calculation (typically DFT) for each snapshot. Other cases where classical out of the box forcefields may diverge from ab-initio calculations is in systems with a presence of metals — the high electronic density in metallic atoms has a non-trivial influence on the distribution of charges on its surroundings that are neglected in normal MD forcefields and instead treated as point charges. 

The significance in using out of the box forcefields for nanomaterials is partly due to the interesting electronic structure of the material in question — which after all is the foundation of why most bear the function that is employed for the task in using them in the field of biological systems. The electronic structure of the nanomaterial we will deal with in this work is highly sensitive to the distribution of functional groups, these have been replicated to the best of our ability with the widely accepted models as derived by experimental work. We see that given this representation of the model of functionalised graphene we arrive at charge distributions from quantum calculations that have a high variance and divergence from out of the box forcefield parameters. 
% ---------------------------
%           Graphene
% ---------------------------

\section{Graphene and graphene derivatives}

\subsection{Physics - invention and application there}
Don't leave this as a subsection -- just using it to guide my writing. A simple introductory paragraph will do.

\subsection{Engineering - desalination and so on}
A nice introduction to your work on GO in solution and what potential it has there.

\subsection{Medical}

% ------------------------------------------------- End intro on Graphene

\subsection{Shortsightedness of graphene research}

\subsubsection{\emph{In vitro} studies neglect \emph{in vivo} effects of bio-incompatibility}
Adsorption of serum proteins will affect fate and toxicity, but this is way beyond the scope of MD. The only advantage of MD is its ability to verify or negate the experimental findings - but I don't see any scientific reason why this would be of any interest (since experiment is scripture), other than elucidating the atomistic drivers for adsorption of hard corona shell. 

The costly lesson to come from MD would be a simple realisation that proteins expose hydrohobic residues that drive further protein-protein interactions, hence compounding the protein adsorption from hard to soft corona. One \emph{could} then derive another expensive lesson that protein size and volume correlate with structure degrees of freedom, therefore a higher likelihood that those larger and more flexible proteins would preferentially form the hard corona layer. 

The alternative approach using MD would be informed by the surface chemistry-driven (primarily charge) binding. Namely, how materials (GO and C2GO) compare in adsorbing different types of target molecules (small peptides, lipids, etc). The loss of detail in this approach would include the macro properties of target molecules; surface shape, charge, structural rigidity are inaccessible and would prove insufficient to infer properties of large proteins, lipid membrane composed of simulated individual lipid molecules and so on. However there is space for a data-informed \emph{and} MD-assisted pipeline, which could accelerate the currently available methods for arriving at novel nanomaterials for any \emph{in vivo} application, while minimising a cost function of interest to the respective research question i.e. adsorption of toxic serum proteins.

\edit{This however will not be covered here and is a research question that can benefit from widely available data of protein structure married with Adaptive Poisson-Boltzmann Solver (APBS) electrostatics calculations or a statistical study of the correlation of amino acid sequence with toxic corona profiles \emph{and} individual amino acid adsorption on graphitic sheets.}

\subsubsection{QM studies neglect surface chemistry of functionalised graphene}
\begin{itemize}
    \item Steric effects and curvature -- charges affected?
    \item Two-phase nature of functionalised graphene -- charge affected?
    \item Edge atoms neglected through implementation of PBC? Especially important if and when dealing with materials such as GO where carboxylic acid can drive function of material when in complex with bio system.
\end{itemize}

\subsubsection{MD studies neglect chemical consequences of heterogeneous material}
This is primarily the charge distribution, however this has an impact on vdW parameters also.

% ---------------------------
%           Outline
% ---------------------------

\newpage
\section{Thesis outline}
\subsection{\edit{leave this til end of this chapter, and write at the end of thesis}}
Of these models, the XXXXXXXXXX model posits a non-random distribution of functional groups, leaving swathes of the nano sheet with sp2 carbon atoms that behave as a conductor, and islands of functionalised graphene where sp3 carbon atoms act as a junction for the flow of electrons (CORRECT?) And would therefore act as an insulator were the entirety of the sheet populated without a guided choice of both ratio of C/O atoms as well as their spatial distribution. (FOR FURTHER REAL SCIENTIFIC EXPLANATION OF THIS AND THE ASSOCIATED EXPERIMENTS, MODELS AND PAPERS SEE ROBBY’S PAPER(S) AND/OR ASK HIM). 

In chapter 2 (ofc not sure yet) we present a comparison between the OPLS and DDEC parameters and their effect on the structure and interaction between graphene-oxide sheets in saline solution. This chapter motivates the necessity for ad-hoc ab-initio forcefield design for nanomaterials (all or specifically those that are non-insulators?) And outlines the effect of approximating an out of the box forcefield for the use in conjunction with biological systems on experimental observables. 

Chapter 3 uses the DDEC forcefield to study the conjunction of nano sheets with serum proteins to learn about their adsorption and the effect of subsequent formation of protein coronas — currently an obstacle to utilising nanomaterials in vivo. The presence of a nanomaterial in the vicinity of serum proteins induces an interaction due to the high binding affinity to engineered nano-materials that were not (naturally or otherwise) designed to operate in a biological system. Such an interaction induces denaturing upon adsorption onto the surface of the nano material — hence causing downstream effects on protein aggregation and inhibited cellular function and cell death. The first layer of adsorbed proteins is known as the hard protein corona.

The choice of nanomaterial drives the components of the hard corona profile and subsequently the toxicity, screening from cellular defences and penetration of particular lipid membranes hence targeting particular cell types such as brain cancer cells. The behaviour driving the function of the nanomaterial at the interface with the protein of interest is happening on an atomistic scale, and therefore require MD simulations to understand the accurate driving forces of the complex at the interface. Such a thorough understanding can therefore enable the quality-by-design production of a novel corona profile — controlled through altering the functional groups on the graphene sheet — to meet a targeted function. Furthermore, simulations should clarify whether the penetration through lipid membranes is mediated by proteins — either adsorbed onto their surface or through the presence of membrane proteins in the cell wall. Differently functionalised sheets have been experimentally shown to target and therefore penetrate different cell types, each of which have a broader toxicity profile which may inhibit their overall efficacy. 

Chapter 4 uses the outcome of this result to gauge the differences in behaviour of cell penetration by comparing MD simulation results to experimental data, this will inform whether or not the corona’s presence in the model is necessary to understand the nanomaterial’s behaviour at the interface with a cell wall. The MD simulations could also, unlike experimental procedures, show if and how a sheet can penetrate a lipid membrane model of a cell type (such as a brain cell model membrane) or a system such as a blood-brain-barrier model (BBB) with or without the presence of adsorbed serum proteins, and wether a pre adsorbed protein can further aid the cellular targeting of the nanomaterial. 

