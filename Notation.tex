% defining commands for creating flowcharts
\tikzstyle{startstop} = [rectangle,rounded corners,minimum width=3cm,minimum height=1cm,text centered,draw=black,fill=gray!30]
\tikzstyle{io} = [trapezium, trapezium left angle=70, trapezium right angle=110, minimum width=3cm, minimum height=1cm, text centered, draw=black]
\tikzstyle{process} = [rectangle, minimum width=3cm, minimum height=1cm, text centered, draw=black, fill=gray!30]
\tikzstyle{decision} = [diamond, minimum width=3cm, minimum height=1cm, text centered, draw=black, fill=yellow!30]
\tikzstyle{arrow} = [thick,->,>=stealth]

% defining python syntax highlighting environment
\DeclareFixedFont{\ttb}{T1}{txtt}{bx}{n}{9} % for bold
\DeclareFixedFont{\ttm}{T1}{txtt}{m}{n}{9}  % for normal

% python syntax colors
\definecolor{deepblue}{rgb}{0,0,0.5}
\definecolor{deepred}{rgb}{0.6,0,0}
\definecolor{deepgreen}{rgb}{0,0.5,0}

% Python style for highlighting
\newcommand\pythonstyle{\lstset{
language=Python,
basicstyle=\ttm,
otherkeywords={self},             % Add keywords here
keywordstyle=\ttb\color{deepblue},
emph={MyClass,__init__},          % Custom highlighting
emphstyle=\ttb\color{deepred},    % Custom highlighting style
stringstyle=\color{deepgreen},
frame=tb,                         % Any extra options here
showstringspaces=false            %
}}

% python environment and inline
\lstnewenvironment{python}[1][]{
  \pythonstyle
  \lstset{#1}
}{}
\newcommand\pythoninline[1]{{\pythonstyle\lstinline!#1!}}

% mathematical formatting
\newcommand{\Vector}[1]{\vec{#1}}
\newcommand{\Matrix}[1]{\mathbf{#1}}

\newcommand{\dt}{\mathrm{d}t}

\newcommand{\edit}{\textcolor{blue}}
\newcommand{\stolen}{\textcolor{red}}
\newcommand{\duplicate}{\textcolor{teal}} %duplicate w/ methods in paper(s) -- may need further depth in intro

\newcommand{\rates}{F_{\theta}}
\newcommand{\tangent}{T_{\theta}}
\newcommand{\steadystates}{\partial S_{\theta}}

\newcommand{\Det}{\left| \frac{\partial\rates}{\partial u} \right|}
\newcommand{\measure}{\varphi_{\theta}}

\newcommand{\predictions}{\mathcal{P}}
\newcommand{\targets}{\mathcal{D}}
\newcommand{\loss}{L}
\newcommand{\error}{E}
\newcommand{\Reals}{\mathbb{R}}
\newcommand{\steady}{u^*}
\newcommand{\cycle}{\omega}
\newcommand{\jacobian}{\frac{\partial\rates}{\partial u}}
\newcommand{\eigenvector}{\hat{v}_\lambda}
\newcommand{\e}{\mathbb{e}}
\newcommand{\degenerate}{u^{-}}
\newcommand{\Real}{\Re\mathrm{e}}
\newcommand{\Imag}{\Im\mathrm{m}}