\addcontentsline{toc}{chapter}{Appendices}
\appendix

\chapter{Interpretation of Morphogen Gradients by a Bistable Circuit}
\label{appendix:double-exclusive}
\includepdf[pages=1-51, offset=75 -90, scale=0.85, frame,
        clip,trim=20mm 5mm 20mm 15mm,
        pagecommand={}, addtotoc={
                2,section,1,Supplementary Figures,appendix:double-exclusive:figures,
                18,section,1,Supplementary Methods,appendix:double-exclusive:methods,
                19,subsection,2,Differential Equation Models \& Parameter Inference,appendix:double-exclusive:inference,
                40,subsection,2,Bistability Analysis,appendix:double-exclusive:bistability,
                42,subsection,2,Boundary Experiments,appendix:double-exclusive:boundaries,
                50,subsection,2,Models of the Exclusive Receiver Relay Circuits,appendix:double-exclusive:relay},
        addtolist={
                2, figure, {\textit{Supplementary Figure 1}\quad Circuit variants}, fig:double-exclusive:variants,
                4, figure, {\textit{Supplementary Figure 2}\quad Raw timecourse fluorescence traces}, fig:double-exclusive:plate-data,
                15, figure, {\textit{Supplementary Figure 13}\quad Hysteresis flow cytometry experiments}, fig:double-exclusive:flow-hysteresis,
                41, figure, {\textit{Supplementary Figure 25}\quad Bifurcation curves for uniform and protected degradation models}, fig:double-exclusive:degradation-models,
                41, figure, {\textit{Supplementary Figure 26}\quad ifurcation curve insensitivity specific growth rate $\gamma_0$}, fig:double-exclusive:growth-rate-sensitivity
}]{publications/double-exclusive-si.pdf}

\chapter{Parameter Inference with Bifurcation Diagrams}
\label{appendix:inference}
\includepdf[pages=1-6, offset=75 -90, scale=0.85, frame,
        clip,trim=33mm 20mm 33mm 20mm,
        pagecommand={}, addtotoc={
                1,section,1,Bifurcation Diagrams as Tangent Fields,appendix:tangent-fields,
                2,section,1,Bifurcation Measure Properties,appendix:bifurcation-measure,
                3,section,1,Leibniz Rule for Space Curves,appendix:leibniz-rule,
                5,section,1,Application to the Double Exclusive Model,appendix:more-complex-model,
                6,section,1,Extension for Hopf Bifurcations,appendix:hopf-measure},
        addtolist={
                1, figure, {\textit{Supplementary Figure 1}\quad Two implicit surfaces $f_{\theta}(z)=0$ and $g_{\theta}(z)=0$ in $\mathbb{R}^3$ intersecting to form a space curve which is tangent to field $\tangent(z)$ and perpendicular to gradients $\partial_{z}f_{\theta}$ and $\partial_{z}g_{\theta}$}, fig:implicit-surfaces,
                2, figure, {\textit{Supplementary Figure 2}\quad Left/Right : Determinant $\Det$ and tangent field $\tangent(z)$ for the saddle-node/pitchfork models for some set values of $\theta$ revealing that $\Det=0$ defines bifurcations}, fig:determinant-field,
                5, figure, {\textit{Supplementary Figure 3}\quad Bifurcation inference for the \emph{double exclusive reporter}. A. Optimal parameter estimates $\theta^*$ for the targets $\targets=\{1,2\}$ (indicated by yellow lines in panel B) reveal four regions  with two geometrically different regimes: mutual activation (region 1) and mutual inhibition (regions 2-4). B. Example bifurcation diagrams indicate that region 2 has swapped kinetics between $L$ and $T$ to region 3. Region 4 has models with non-zero imaginary parts to eigenvalues indicating damped oscillations (shown in light green).},
                fig:double-exclusive-optima,
                6, figure, {\textit{Supplementary Figure 4}\quad Bifurcation measure $\measure(s)$ and eigenvalues $\lambda(s)$ along the arclength $s$ for two different bifurcation curves demonstrating how the measure detects non-zero imaginary parts $\Imag[\lambda]$ (onset of damped oscillations marked by circle) and sign changes in real parts $\Real[\lambda]$ (Hopf bifurcations marked by stars)},
                fig:hopf-measure
}]{publications/bifurcation-inference-si.pdf}
