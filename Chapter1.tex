\chapter{Introduction \& Motivation}
\label{chapter:introduction}
%
\epigraph{\textit{I wish to God these calculations had been executed by steam.}}{Charles Babbage}
The advent of the modern digital computer, as formalised by Alan Turing,\cite{Turing1937} ignited the field of computational physics, aided by preexisting theoretical formulations of algorithms. Starting from the first experiments with Monte Carlo (MC) simulations in the 1930s by Fermi and the formulation of the Markov-Chain Monte Carlo (MCMC) technique by Ulam in the 1940s, von Neumann programmed the 18,000 vacuum-tube Electronic Numerical Integrator and Computer (ENIAC) computer to investigate neutron diffusion in fissionable materials.\cite{metropolis1987beginning} This success paved the way for the integration of Newton's equations of motion to compute the time evolution of a many-body system.\\

\section{Current state-of-the-art in biomolecular simulations}
\label{sec:sota_bio_sims}
\subsection{Groundbreaking simulations}

\subsection{Bespoke forcefield parametrisation}
\label{sec:ff_devel}

\subsubsection{OpenFF initiative}

\subsubsection{Gaussian process regression}

\subsubsection{Neural network forcefields}

\subsubsection{QUBE forcefield}

\subsubsection{Outlook}

%%%%%%%%%%%%%%%%%%%%%%%%
%     Architectures
%%%%%%%%%%%%%%%%%%%%%%%%
\subsection{State of the art architecture}

\subsubsection{GPU acceleration}

\subsubsection{ASIC}

\subsubsection{FPGA}

%%%%%%%%%%%%%%%%%%%%%%%%
%    Finally, make 
%     your case 
%%%%%%%%%%%%%%%%%%%%%%%% 
\section{Motivation}
