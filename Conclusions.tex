\chapter{Conclusions}
\label{chapter:conclusions}

%\epigraph{\textit{``Nothing more will I teach you today. Clear your mind of questions.”}}{Yoda}
\epigraph{\textit{I am leaving the regions of fact, which are difficult to penetrate, but which bring in their train rich rewards, and entering the regions of speculation, where many roads lie open, but where a few lead to a definite goal.}}{William Ramsay}

Using the methods described throughout this thesis, I have attempted to employ accurate multiscale modelling to address problems within the domain of biology where conventional investigation techniques may be less suitable. This work collectively shows the importance of a multipronged approach to accurate modelling, namely the interdependence of the correct characterisation of structure with the accuracy of the theoretical treatment of a model. \\

I have used density functional theory as a basis to derive a quantum mechanical forcefield for accurate classical molecular dynamics simulations of a graphitic nanomaterial composed of around 1000 atoms. Using this, we investigated the impact of the correlated semi-ordered functionalisation as well as the bespoke forcefield parameters on the nanomaterial's interactions with an ionic solution and its chaotropic potential. This study illustrated the capacity for extrapolating bespoke forcefield parametrisation methods to very large molecular systems, without the need to sacrifice accurate characterisation, which we show is principal to reproducing both experimental and state of the art \textit{ab initio} measurements. Having developed an accurate forcefield model for the system, it is extendable to MD simulations of hundreds of thousands of atoms. \\

Using the accurate structural model of graphitic materials, we used our model to address an open question relating the atomistic interactions at the bio-nano interface. The formation of the protein corona is an obstacle to effectively translating the technological advancements of nanomaterials to biotechnology. The dynamic character of the protein corona has made it a challenging problem to tackle using conventional computational and experimental methods alike. The investigation reported in this thesis utilised numerous analysis techniques to form a collective analysis pipeline to unpick the impact of nano-functionalisation on adsorbed protein structure. Using our results, we were able to make sense of experimentally observed behaviours of the much larger protein corona from a single adsorbed common serum protein. These findings included the way in which some nanomaterial functional groups induce the adsorbed protein to denature its tertiary structure and form binding motifs for protein aggregation; this instigates the formation of a protein corona as observed in experiment. Furthermore, using our analyses we explained the experimentally observed contrast in cellular uptake between different nanomaterial-corona complexes, differing only by functionalisation type. Our analyses showed the impact of conserving functionally important sequences in the protein on the cellular uptake of the nanomaterial-protein complex. Much of the observed behaviour was sensitive to the nanomaterial characterisation, again reinforcing the importance of accurate modelling to associate molecular modelling with experiment.\\

The accurate atomistic modelling of a protein active site, aided by data analysis techniques to unscramble and interpret the dynamics, formed the basis for understanding the role of a catalytic dyad in the function of the SARS-CoV-2 main protease. The protease plays the role of cleaving the viral polyprotein, inhibiting its function therefore disables subsequent viral replication. Unlike the now prevalent preventative immunogenic approaches, protease inhibitors do not require an immunogenic response and can be used to treat both severely afflicted or immunocompromised patients. Unlike many computational approaches to drug discovery, MD simulations clarify the change in dynamics of the protein due to the influencing presence of an inhibitor. Without the previous identification of a potent inhibitor for the similar active site of the SARS-CoV-1 main protease, we would not have been able to unpick the mechanism of the catalytic dyad disruption and use it to calculate the free energy of binding. Since that work was written, the continued efforts to identify protease inhibitors through X-ray crystallography screening have observed numerous ligands that disrupt the His41-Cys145 catalytic dyad. Accelerating the rare-event sampling of the dyad disruption is achieved through the use of Metadynamics simulations, which is completely transferable and open to use by other MD-based projects. This can be used to sample the catalytic dyad disruption within computationally feasible timescales, in order to perform high-throughput screening for as many molecules as possible while retaining the advantages of modelling a dynamic target. \\

The use of electronic structure calculations that account for strong electronic correlations have mostly been reserved for small molecules or periodic systems. We have translated the hybrid DFT+DMFT method to a molecular system to study the active site of the hemocyanin oxygen-transporting protein found in the hemolymph of some invertebrates. It is the first application of cluster DMFT to a biological system, where it identifies the electronic structure of an uncommon open-shell singlet ground state. The singlet ground state is a quantum-entangled superposition of two localised magnetic moments on two distant copper atoms in the hemocyanin active site. This investigation details the function of a superexchange mechanism where electron hopping between the copper $d$-shells is mediated by the bridging dioxygen ligand $p$-orbitals. Its role in reversible oxygen binding is inevitable but is yet to be described using this new information. \\

This work stresses that it is biology that defines the warranted accuracy with which it is to be treated, in line with experimental results to which we return for verifiability. In the pursuit of accuracy, no single theoretical approach can be used to tackle an array of questions aimed at the same class of biomolecular systems. The progress of interdisciplinary collaboration, the advancement of open-source scientific software and increasing experimental data all coalesce into a flourishing environment for collaborative scientific research to advance the multiscale modelling of biomolecular systems and their interface with state-of-the-art nanomaterials. 

\section{Future work}

Having demonstrated the implementation of theoretical approaches to the biological domain in different applications, an extension of this work would build on the most significant divergence from experiment in preexisting methods. The parametrisation of forcefield parameters for graphitic materials is well suited to describing the interfacial properties of a complex chemical environment where strong electrostatic interactions drive the interfacial phenomena. As such, the model could be extended to study the interactions with lipid bilayer systems for which there is abundant demand for multiscale modelling. Furthermore, the availability of linear-scaling electronic structure calculations and forcefield parametrisation software can be used to extend the accuracy to much larger systems in the biological domain. Unlike the predominant application of the interface between biomolecular systems and bespoke forcefield parametrisation of small molecules in the drug discovery domain, the work in this thesis describes its extension to the rapidly growing field of biotechnology. \\

The accurate characterisation and function of the SARS-CoV-2 main protease active site as well as the acceleration of rare-event sampling has been described. This particular work can be extended into an automated pipeline for high-throughput screening of inhibitor ligands. Given the time and computational limitations, this extension remained outside our capacity but nonetheless forms a qualitative proof of principle of how accurate modelling can inform drug delivery efforts for a subset of inhibitor ligands that require the rare-event disruption of the catalytic dyad. \\

The strongly correlated ground state of the hemocyanin active site was resolved using DFT+DMFT and can be extended to other metalloproteins with quantum-mechanically driven biological function. Interestingly, the accurate characterisation of the electronic structure of the hemocyanin active site following this work was extended to bio-mimetic applications in the selective catalytic reduction of nitrogen-oxide pollutants.\cite{chen2019comparative} \\

The field of biotechnology will necessitate the accessible coherent research from both computational and experimental perspectives. Accurate tools are therefore an invaluable requisite for the advancement of the field where the bio-nano interface is concerned. The concomitant advancement of computational architectures and \textit{good} scientific software will improve accessibility to flawlessly research and inform the design of biotechnological solutions. 
