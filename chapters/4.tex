\chapter{Parameter Inference with Bifurcation Diagrams}
\label{chapter:inference}
\begin{music}
    \parindent10mm \instrumentnumber{1} \setstaffs1{1} 
    \generalmeter{\meterfrac44} \generalsignature{2}
    \startextract
            \notes \ql j \ql i \Qqbl ieji \en
        \bar \zw{m*} \bar 
            \notes \ql i \qu h \Qqbu hdih \en
        \bar \zw{l*} 
    \zendextract
\end{music}
\epigraph{\textit{the melody that will draw you into the infinite darkness}}{Nocturne of Shadow --- Ocarina of Time}

\section{Preface}
\subsection{Problem Statement \& Context}

This chapter focuses on the problem of looking for parameter regimes for dynamical systems that result in bifurcations. This has been coined as \emph{inverse bifurcation analysis} \cite{Lu2006InverseSystems}. Formally we are looking for parameters $\theta$ for which the equation \eqref{eq:differential-equations} has at least one fixed point for which one of the bifurcation criteria (see Section \ref{section:bifurcation-analysis}) is satisfied. Furthermore we would like to create a specified type and number of bifurcations to be placed along control condition $p$.

Inverse bifurcation analysis becomes relevant to biomedical researchers in settings where there is a design goal to engineer an organism with a distinct phenotype (as discussed in Chapter \ref{chapter:introduction}).
Such a phenotype can be a the \emph{double exclusive reporter} investigated in the interdisciplinary collaboration in synthetic developmental biology presented in Chapter \ref{chapter:double-exclusive}. This \emph{E. Coli} phenotype exhibits a cusp bifurcation leading to a bistable response with respect to two input signals as shown in Figure \ref{fig:double-exclusive:bistability}. The engineering goal could be a phenotype with a specific cell cycle \cite{Attila2008,Conrad2006BifurcationClock} which involves the positioning of \emph{Hopf bifurcations} that mark the onset of oscillations in protein concentrations from stable equilibria. Finally the design of self-organised patterns such as stripes and spots in mammalian coat patterns could be another engineering goal. According to a long standing mathematical hypotheses \cite{Turing1952} that underpins developmental mechanisms this may involve the search for \emph{Turing bifurcations}.

Before we get carried away and think that with the right tooling we could genetically control the length scale of spots or stripes on cats, we must remind ourselves that \emph{in vivo} gene regulatory networks mostly consist of unknown and experimentally inaccessible parameters. Furthermore, we find that usually there exist multiple equally valid models that describe the observed behaviour, and so turn to \emph{model selection} methods. Even if we had an accurate and unique model to describe an organism, generic tools for \emph{inverse bifurcation analysis} are limited as we've explored in Chapter \ref{chapter:background} and experienced in practice in Chapter \ref{chapter:double-exclusive}. In an attempt to address these limitations, the incorporated publication (sections \ref{inference:abstract} -- \ref{inference:impact}) focuses on the design of systems of ordinary differential equations with \emph{pitchfork} and \emph{saddle-node} bifurcation diagrams. Although the publication focuses on parameter synthesis for a subset of bifurcations, steps towards \emph{Hopf} bifurcations are made (see Appendix \ref{appendix:hopf-measure}), and the approach lays foundations for differentiable optimisation methods that leverage bifurcation theory. A view towards how this approach can be used for the design of \emph{Turing patterns} and model selection is discussed in concluding Chapter \ref{chapter:conclusions}.

\subsection{Contributions}

\textbf{Grisha Szep} prepared the manuscript, designed the cost function, derived mathematical results, wrote and released the Julia package under the supervision of \textbf{Neil Dalchau} and \textbf{Attila Czikasz-Nagy}.

\includepdf[pages=1-11, offset=75 -95, scale=0.85, frame,
        clip,trim=31mm 21mm 31mm 21mm,
        pagecommand={}, addtotoc={
        1,section,1,Abstract,inference:abstract,
        1,section,1,Introduction,inference:introduction,
        2,subsection,2,Preliminaries,inference:preliminaries,
        4,section,1,Proposed Method,inference:method,
        4,subsection,2,Semi-supervised Cost Function,inference:cost,
        5,subsection,2,Differentiating the semi-supervised cost function,inference:derivatives,
        6,section,1,Experiments \& Results,inference:results,
        6,subsection,2,Minimal Models,inference:minimal,
        6,subsection,2,Genetic Toggle Switch,inference:genetic,
        7,subsection,2,Complexity,inference:complexity,
        9,section,1,Conclusion \& Broader Impact,inference:impact,
        10,section,1,References,inference:references},
    addtolist={
        3, figure, {\textit{Fig. 1}\quad Illustration of bifurcation diagrams for minimal models of bifurcations. A. Saddle-node bifurcations arise for $\rates(u,p) = p + \theta_{1}u+\theta_{2}u^3$ when $\theta = (\frac{5}{2},-1)$. B. Pitchfork bifurcations arise for $\rates(u,p) = \theta_{1} + p u+\theta_{2}u^3$ when $\theta=(\frac{1}{2},-1)$. Targets are illustrated by light yellow vertical lines. Bifurcation curves are shown as solid blue and red lines, with lighter shades indicating the determinant crossing zero at locations $\predictions(\theta)$ giving rise to unstable solutions.}, fig:inference:minimal-models,
        4, figure, {\textit{Fig. 2}\quad Bifurcation measure $\measure(s)$ and determinant $\Det$ along the arclength $s$ of two different bifurcation curves demonstrating how maximising the measure along the curve maintains the existing bifurcation marked by a circle, while encouraging new bifurcations marked by stars.}, fig:inference:measure,
        6, figure, {\textit{Fig. 3}\quad Saddle-node $\rates(u,p) = p + \theta_{1}u+\theta_{2}u^3$ and pitchfork $\rates(u,p) = \theta_{1} + u p +\theta_{2}u^3$ optimised with respect to $\theta$ so that predicted bifurcations $\predictions(\theta)$ match targets $\targets$ in control condition $p$. The right panel shows bifurcations diagrams for the three optimal $\theta^*$ marked by stars on the left panel. The optimisation trajectories in white follow the gradient of the cost, approaching the black lines of global minima in the left panel}, fig:inference:minimal-models:results,
        7, figure, {\textit{Fig. 4}\quad Bifurcation inference for the two-state model (11). A. Optimal parameter estimates $\theta^*$ for the targets $\targets=\{4,5\}$ reveal two clusters of qualitatively different regimes: mutual activation ($a_1 < 1$; cluster 1) and mutual inhibition ($a_1 > 1$; cluster 2). B. Example bifurcation diagrams indicate positively and negatively correlated dependencies between the two model states, as a function of the control condition.}, fig:inference:two-state-optima,
        8, figure, {\textit{Fig. 5}\quad A. Execution time (time to calculate cost gradient) with respect to states $N$. B. Convergence times (the time it takes to find and match a bifurcation to within 1\% of a specified target) with respect to the number of parameters $M$, comparing against a gradient-free approach: Nelder-Mead. Calculations were performed on an Intel Core i7-6700HQ CPU @ 2.60GHz x 8 without GPU acceleration.}, fig:scaling
}]{publications/bifurcation-inference.pdf}

\section{Afterword}