\chapter{Parameter Inference with Bifurcation Diagrams}
\label{chapter:inference}
\begin{music}
    \parindent10mm \instrumentnumber{1} \setstaffs1{1} 
    \generalmeter{\meterfrac44} \generalsignature{2}
    \startextract
            \notes \ql j \ql i \Qqbl ieji \en
        \bar \zw{m*} \bar 
            \notes \ql i \qu h \Qqbu hdih \en
        \bar \zw{l*} 
    \zendextract
\end{music}
\epigraph{\textit{the melody that will draw you into the infinite darkness}}{Nocturne of Shadow --- Ocarina of Time}

\section{Preface}
\subsection{Problem Statement \& Context}

This chapter focuses on the problem of looking for parameter regimes for dynamical systems that result in bifurcations. This has been coined as \emph{inverse bifurcation analysis} \cite{Lu2006InverseSystems}. Formally we are looking for parameters $\theta$ for which the equation \eqref{eq:odes} has at least one fixed point for which one of the bifurcation criteria (see Section \ref{section:bifurcation-definitions}) is satisfied. 

Inverse bifurcation analysis becomes relevant to biomedical researchers in settings where there is a design goal to engineer an organism with a distinct phenotype (as discussed in Chapter \ref{chapter:introduction}).
Such a phenotype can be a the \emph{double exclusive reporter} investigated in the interdisciplinary collaboration in synthetic developmental biology presented in Chapter \ref{chapter:double-exclusive}. This \emph{E. Coli} phenotype exhibits a cusp bifurcation leading to a bistable response with respect to two input signals as shown in Figure \ref{figure:double-exclusive:bistability}. The engineering goal could be a phenotype with a specific cell cycle \cite{Attila2008,Conrad2006BifurcationClock} which involves the positioning of \emph{Hopf bifurcations} that mark the onset of oscillations in protein concentrations from stable equilibria. Finally, that which possibly demonstrates the most engineering prowess, is the design of self-organised patterns such as stripes and spots in mammalian coat patterns. According to a long standing mathematical hypotheses \cite{Turing1952} that underpins developmental mechanisms this may involve the search for \emph{Turing bifurcations}.

Before we get carried away and think that with the right tooling we could genetically control the length scale of spots or stripes on cats, we must remind ourselves that \emph{in vivo} gene regulatory networks mostly consist of unknown and experimentally inaccessible parameters. Furthermore, we find that usually there exist multiple equally valid models that describe the observed behaviour, and so turn to \emph{model selection} methods. Even if we had an accurate and unique model to describe an organism, generic tools for \emph{inverse bifurcation analysis} are limited; see Chapter \ref{chapter:background} for details. In an attempt to address the limitations encountered in Chapter \ref{chapter:double-exclusive} the published work in this chapter focuses exclusively on \emph{pitchfork} and \emph{saddle-node} bifurcations. Although the publication focuses on parameter synthesis for a subset of bifurcations, the approach lays foundations for differentiable optimisation methods that leverage bifurcation theory. A view towards how this approach can be used for the design of \emph{Turing patterns} and model selection is discussed in concluding Chapter \ref{chapter:conclusions}.

\subsection{Extracting Bifurcations from Single Cell Data}

The publication in this chapter assumes that the bifurcation data $\targets$ with respect to the experimental control parameter $p\in\Reals$ is known or already calculated from raw data. In the following sections we focus on methods for extracting bifurcation data $\targets$ from single cell trajectories and flow cytometry measurements.

The decision to focus on single cell trajectories and flow cytometry came from the limitations of using microplate data in Chapter \ref{chapter:double-exclusive}. The model parameters $\theta$ were estimated using a hierarchical monte-carlo approach and time-course fluorescence microplate measurements (details of which can be found in Appendix \ref{appendix:double-exclusive:inference}). The time-courses include information about dynamical transients and colony growth in liquid culture as shown in Figure \ref{fig:microplate-data}. The desired cusp bifurcation, however, lives in state-space rather than the time-domain. The disconnect between the domain that the data lives in and the domain of the design goals poses the risk of over-fitting the model on undesired information that exists in the data domain. 

It is not possible to observe the cusp bifurcation in microplate data, due to the averaging of signals originating from heterogeneous cell populations. Instead, the cusp bifurcation can be observed in flow cytometry measurements of colonies in exponential phase (Supplementary Figure \ref{figure:double-exclusive:flow-hysteresis}) and microfluidic fluorescence microscopy data (Figure \ref{figure:double-exclusive:bistability}c) where computations on single-cell trajectories reveal the hysteresis loop which must necessarily accompany the cusp. 

\subsection{Single Cell Trajectories}
% Non-parametric ___ using Gaussian Process regression
\label{section:field-inference}

In this early days of this thesis, we investigated whether it was possible to transform the time-domain data into state-space. This approach, and related works, are discussed in this section and can in principle be used with the microfluidic fluorescence microscopy data for parameter inference.

Consider we are given $K$ cell trajectories $\mathcal{D}_1$, $\mathcal{D}_2$ ... $\mathcal{D}_K$, each containing $N$ noisy observations of the state of the cell. Let the cell state be represented by state vector $u(t)\in\Reals^N$ which is hypothesized to obey a set of ordinary differential equations of the form \eqref{eq:odes}. Instead of integrating the equations \eqref{eq:odes} we would find an estimate for the derivative of the trajectories $\hat{f}$.

This is known as the \textit{smoothing} step \cite{Gugushvili2012Smoothing} should be done using unsupervised methods, for example with Gaussian Process Regressors \cite{Seeger2004GaussianLearning.} as shown in in Figure \ref{fig:inferred-cycles}. This requires the inversion of an $K'\times K'$ data matrix where $K':=\sum_k |\mathcal{D}_k|$ is the total number of trajectory data points. This has a computational complexity $K'^3$ which is only tractable with sparse datasets.

Let the region $\partial\mathcal{D}$ be a boundary defined by the Delaunay tessellation of the input data. Let us define the estimate $\hat f$ only within the region $\partial\mathcal{D}$ so that there are no extrapolation artefacts. For the Gaussian Process approach the estimate would be
\begin{equation}
    \hat{f}(u)\sim
        \mathcal{N}(\,\mu(u) ,\Matrix{\Sigma}(u)\,)
    \quad\mathrm{for}\quad u\in\partial\mathcal{D}
\end{equation}
\noindent where at any given state $u$ the field estimate $\hat{f}$ is generated by Gaussian distributions of mean vector $\mu$ and covariance matrix $\Matrix{\Sigma}$.Solving for these requires a choice of matrix-valued kernel function $\Matrix{K}(u,v)$ which encodes our knowledge about the local structure of the field. Sophisticated kernels for learning vector fields exist \cite{Fuselier2017ADecompositions} for decomposing fields in conservative and solenoidal components, which aid in localising fixed points and cycles.

The simplest choice of kernel assumes the components are independent and have a finite correlation length $\gamma$, such as Gaussian radial basis functions. Here $\Matrix{I}$ is the identity matrix and the hyperparameter $\gamma$ has to be optimised.
\begin{equation}
    \Matrix{K}(\Vector{u},\Vector{v}) = \Matrix{I}\,\mathbb{e}^{-\gamma|\Vector{u}-\Vector{v}|^2}
\end{equation}

The second step is called \textit{matching} where the estimated field $\hat{f}$ is used as an optimisation target against some parametrised function $\rates$ with unknown parameters $\theta$.

In our setting we would like to match the geometry of the field but not its magnitude; in this sense we are focusing on the qualitative aspects of the dynamics of a set of differential equations, rather than the quantitative dynamics or kinetics. This could be achieved with the following objective function
\begin{equation}
    \mathcal{L}(\theta|\mathcal{D}) := e^{-\frac{\hat{f}\cdot\rates}
    {|\hat{f}||\rates|}}
\end{equation}
\noindent where the cost is minimal when the data derivative $\hat{f}$ and the parametrised model $\rates$ point in the same direction and maximal when they point in opposing directions.

\begin{Figure}
    \includegraphics[width=125mm]{figures/cycle-2.png}
    \includegraphics[width=125mm]{figures/cycle-1.png}
    \caption{Gaussian process regressors estimating derivative of the trajectories $\hat{f}$ from example trajectory datasets $\mathcal{D}_1$ ... $\mathcal{D}_K$ with varying signal to noise ratios. Interpolation error $E$ is shown as a heatmap; extrapolation fails}
    \label{fig:inferred-cycles}
\end{Figure}

\subsection{XXX}

Although we are getting close to focusing on qualitative features of a model, this objective function is still sensitive to the locations and shapes of fixed point and limit cycles. What if we cared about even higher-level features such as the number of fixed points? Or perhaps whether a system oscillates or not? This is where the language of bifurcation theory described in Chapter \ref{chapter:background} is optimally suited for this task, but fist we need to discuss how to set up experiments to detect bifurcations from flow cytometry data.

\begin{itemize}
    \item \todo[inline]{Vector field estimates too noisy to get bifurcation points?}
    \item \todo[inline]{Divergence and the stability of fixed points}
    \item \todo[inline]{Curl and limit cycles. How does this relate to hopf example \ref{fig:inferred-cycles}}
    \item \todo[inline]{Do we need a bistable example?}
\end{itemize}

The accuracy of the cell trajectories is limited by cell segmentation and tracking algorithms. Initial investigations into this approach also suggested that trajectories need to be of sufficient temporal resolution and sampled from a wide variety of initial conditions. Such data is not widely available and ultimately we decided to focus on a method that could be used with a well-known workhorse in biomedical research: flow cytometry.

\subsection{Flow Cytometry}
\label{section:flow-calculations}

Suppose we would like to detect a cusp bifurcation in a cell population with respect to two experimental control conditions $p_1,p_2\in\Reals$. We set up a serial dilution along the rows for $p_1$ and along the columns for $p_2$ in two 96-well plates. Cells are seeded and allowed to grow in exponential phase in a finite concentration of one of the two conditions. This is done for both conditions, resulting in two cell populations: $p_1$-primed cells and $p_2$-primed cells. The priming concentrations must be chosen sufficiently high so that they lie either side of the cusp as shown in Figure \ref{fig:cusp-sampling}. The two populations are transferred into separate 96-well plates containing the dilutions of $p_1,p_2$, allowed to grow in exponential phase and then transferred into the flow cytometer.

\begin{Figure}
    \missingfigure[figwidth=10cm, figheight=10cm]{}
    \caption{Primed cells transferred into a dilution of $p_1$ and $p_2$ in a 96-well plate sample different parts of the hysteresis loop, revealing the cusp bifurcation}
    \label{fig:cusp-sampling}
\end{Figure}

After gating live singlet cells and applying relevant compensation and auto-fluorescent normalisation, a figure similar to Supplementary Figure \ref{figure:double-exclusive:flow-hysteresis} can be produced to qualitatively assess if the cusp was sufficiently well sampled.

To quantify bistability $\beta_{P|Q}(p)$ in each well $p=(p_1,p_2)$ we could calculate the separation or overlap between the fluorescence distributions of the two primed populations $P(x|p)$ and $Q(x|p)$. Here $x\in\Reals^{F}$ is an $F$-dimensional fluorescence vector. The simplest calculation is taking the difference between some summary statistic, such as mean or median, between the two distributions in each well. For example, using the mean we write
\begin{align}
    \beta_{P|Q}(p):=\left|
    \int x P(x|p)\mathrm{d}x
    -
    \int x Q(x|p)\mathrm{d}x
    \right|
\end{align}
This approach breaks down if multi-modal distributions exist in the data. This would be the case if something happened to prevent a subpopulation of cells to switch from one state another other. Reasons for this could include too much cell burdon or not enough time given for cells to reach a steady state. In this case, quantifying the efficiency of switching from either side of the cusp could be a quantity of interest.

The set of limit points that define the shape of the cusp can be extracted from one of the contours of the bistability measure
\begin{align}
    \targets = \{  p : \beta_{P|Q}(p)=c \} 
\end{align}
where $c\in\Reals$ is chosen such that contour approximates the limit points between the bistable and monostable regions in $p$.

\begin{itemize}
    \item \todo[inline]{quantify efficiency of switching? data still available}
    \item \todo[inline]{Mention limitation: detecting hopf bifurcation not possible}
\end{itemize}

\subsection{Contributions}

\textbf{Grisha Szep} prepared the manuscript, designed the cost function, derived mathematical results, wrote and released the Julia package under the supervision of \textbf{Neil Dalchau} and \textbf{Attila Czikasz-Nagy}.

\includepdf[pages=1-11, offset=75 -95, scale=0.85, frame,
        clip,trim=31mm 21mm 31mm 21mm,
        pagecommand={}, addtotoc={
        1,section,1,Abstract,inference:abstract,
        1,section,1,Introduction,inference:introduction,
        2,subsection,2,Preliminaries,inference:preliminaries,
        4,section,1,Proposed Method,inference:method,
        4,subsection,2,Semi-supervised Cost Function,inference:cost,
        5,subsection,2,Differentiating the semi-supervised cost function,inference:derivatives,
        6,section,1,Experiments \& Results,inference:results,
        6,subsection,2,Minimal Models,inference:minimal,
        6,subsection,2,Genetic Toggle Switch,inference:genetic,
        7,subsection,2,Complexity,inference:complexity,
        9,section,1,Conclusion \& Broader Impact,inference:impact,
        10,section,1,References,inference:references},
    addtolist={
        3, figure, {\textit{Fig. 1}\quad Illustration of bifurcation diagrams for minimal models of bifurcations. A. Saddle-node bifurcations arise for $\rates(u,p) = p + \theta_{1}u+\theta_{2}u^3$ when $\theta = (\frac{5}{2},-1)$. B. Pitchfork bifurcations arise for $\rates(u,p) = \theta_{1} + p u+\theta_{2}u^3$ when $\theta=(\frac{1}{2},-1)$. Targets are illustrated by light yellow vertical lines. Bifurcation curves are shown as solid blue and red lines, with lighter shades indicating the determinant crossing zero at locations $\predictions(\theta)$ giving rise to unstable solutions.}, figure:inference:minimal-models,
        4, figure, {\textit{Fig. 2}\quad Bifurcation measure $\measure(s)$ and determinant $\Det$ along the arclength $s$ of two different bifurcation curves demonstrating how maximising the measure along the curve maintains the existing bifurcation marked by a circle, while encouraging new bifurcations marked by stars.}, figure:inference:measure,
        6, figure, {\textit{Fig. 3}\quad Saddle-node $\rates(u,p) = p + \theta_{1}u+\theta_{2}u^3$ and pitchfork $\rates(u,p) = \theta_{1} + u p +\theta_{2}u^3$ optimised with respect to $\theta$ so that predicted bifurcations $\predictions(\theta)$ match targets $\targets$ in control condition $p$. The right panel shows bifurcations diagrams for the three optimal $\theta^*$ marked by stars on the left panel. The optimisation trajectories in white follow the gradient of the cost, approaching the black lines of global minima in the left panel}, figure:inference:minimal-models:results,
        7, figure, {\textit{Fig. 4}\quad Bifurcation inference for the two-state model (11). A. Optimal parameter estimates $\theta^*$ for the targets $\targets=\{4,5\}$ reveal two clusters of qualitatively different regimes: mutual activation ($a_1 < 1$; cluster 1) and mutual inhibition ($a_1 > 1$; cluster 2). B. Example bifurcation diagrams indicate positively and negatively correlated dependencies between the two model states, as a function of the control condition.}, figure:inference:two-state-optima,
        8, figure, {\textit{Fig. 5}\quad A. Execution time (time to calculate cost gradient) with respect to states $N$. B. Convergence times (the time it takes to find and match a bifurcation to within 1\% of a specified target) with respect to the number of parameters $M$, comparing against a gradient-free approach: Nelder-Mead. Calculations were performed on an Intel Core i7-6700HQ CPU @ 2.60GHz x 8 without GPU acceleration.}, figure:scaling
}]{publications/bifurcation-inference.pdf}