\title{Inferring bifurcations\\between phenotypes}
\author{Grisha Szep}

\department{Randall Division of Cell \& Molecular Biophysics}
\sponsor{Microsoft Research Cambridge}

\date{October 29, 2021}
\maketitle
% \makedeclaration

\begin{abstract} % 5000 word limit
\vspace{-3em}
    
    The gene-expression history of a cell together with its environment determine the its phenotype. The phenotype is an inherently qualitative state, deduced by relative biochemical measurements, increasingly collected with high-throughput single cell methods. In synthetic biology phenotypes are designed, often with qualitative constraints in mind that enforce a desired behaviour with respect to environmental conditions. The design of phenotypes can be cast as a problem in \emph{inverse bifurcation analysis} when cells are modelled using differential equations. Recently, an emerging discipline called \emph{scientific machine learning} (SciML) has been pushing for differentiability within simulations and structure-informed models. Bifurcation analysis, however, is yet to benefit from differentiability. In this thesis, bifurcation theory is leveraged to address questions in pattern formation with synthetic \emph{E. Coli}. The inter-disciplinary collaboration gave rise a novel approach for estimating the parameters of differential equations. The emergent picture suggests that qualitative observations should be learned with bifurcations prior to any attempt at learning quantitative details that may be subject to inter-experiment variability. Concurrently, a focus on flow cytometry, one of the most abundant methodologies in which gating strategies are used to annotate cell phenotypes, gave rise to \emph{FlowAtlas.jl}: a tool for navigating high-dimensional giga-scale cytometry in settings where manual annotation by domain experts becomes unfeasible. The ideas in this thesis are valuable to those wanting to build interactive and differentiable \emph{design--learn} workflows for biomedical research.

\end{abstract}
\newpage
\clearpage
\thispagestyle{empty}
\noindent Blank pages are the worst.\\
They impose a glaring responsibility onto someone\\
to fill it with meaningful content.\\
Much like other starting points:
a new job, a marble stone, empty land;\\
all futures that tower over you, forcing the question:\\
\emph{"do I really want to do this?"}
\vspace{7pt}\\
Blank pages are the worst.\\
They reflect a glaring light upon you,\\reflecting the messages that would otherwise happily bounce around in the ether along with memories, desires, unfinished projects, and other\\
cacophonous bullshit in your head.
\vspace{7pt}\\
A whole page!\\
Rejoice at the etchings of achievement.\\You've started now so don't give up, or stop: you'll look stupid. Now make sure you go back and edit the previous page so that others will not know how stupid you are.
\vspace{7pt}\\
Edit it. Edit it. Edit it.\\
Edit it into oblivion until you can't recognise whether you are editing the page or yourself. May the tessellating thought loops tighten to the point where they swallow you whole.
\vspace{7pt}\\
Want to know what is worse than a blank page?
Choosing an end.\\End that job that you hate, end your addictions, end your malpractice while trying to keep yourself intact, since maintaining a cohesive narrative will keep you from going batshit.
\vspace{7pt}\\
Like parents who can't leave their children alone,\\ we find it hard to let go of something we've worked on for so long.\\
But as you turn each page, so does fate.\\Look back upon the emptiness that stirred much distaste,
and settle.\\
Settle now that it is over; as all things will be one day
\clearpage

\begin{acknowledgements}
\vspace{-3em}
    The past four years of my life have been full of excitement, creativity and exploration, none of which would have been possible without the support of Attila and Neil. Their mentorship and guidance has been a shining example of leadership and whose qualities I will take with me into future roles. Thank you to my colleagues and friends at Microsoft Research Cambridge, whose diverse projects in biotechnology captured my imagination. I would like to acknowledge Valerie Coppard, who was an absolute pleasure to work with and introduced me to Joanne Jones' lab at Cambridge University, catalysing the \emph{FlowAtlas.jl} project. Her enthusiasm was infectious and I am grateful her for sharing her knowledge in biology. A special thank you to my friends and partners in science: Matilda Peruzzo, Silvia Caballero Mancebo at the Institute of Science and Technology Austria and Mohamed Ali Al-Badri at UCL. Thank you for believing in me and keeping me sane during the isolation of the pandemic. I owe a great deal to the Physics department at King's College London; Lev Kantorovich, Alessandro de Vita and Riccardo Sapienza for opportunities during my undergraduate years and Paul Le Long for always making me feel welcome and at home for years after graduation. I am thankful to my friends at Burnt Umber who run a cozy and welcoming cafe in Hackney Wick. I wrote a large chunk of my thesis there, felt supported, and cared for during difficult times. I am grateful to my therapist, Disree Shaw, for teaching me what it feels like to be comfortable in the discomfort. I am grateful to my partner, Fraser Wing, for his love and support. I dedicate this thesis to my mother, whose love and strength I will always remember, and to my father, who has done an excellent job at raising a doctor.

\end{acknowledgements}

%%%%%%%%%%%%%%%%%%%%%%%%%%%%%%%%%%%%%%%%
%%%%%%%%%%%%%%%%%%%%%%%%%%%%%%%%%%%%%%%%
%%%%%%%%%%%%%%%%%%%%%%%%%%%%%%%%%%%%%%%%
%%%%%%%%%%%%%%%%%%%%%%%%%%%%%%%%%%%%%%%%
%%%%%%%%%%%%%%%%%%%%%%%%%%%%%%%%%%%%%%%%

\setcounter{tocdepth}{2} 
% Setting this higher means you get contents entries for
%  more minor section headers.

\tableofcontents
\listoffigures