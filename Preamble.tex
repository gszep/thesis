\title{Inferring bifurcations\\between phenotypes}
\author{Grisha Szep}

\department{Randall Division of Cell \& Molecular Biophysics}
\sponsor{Microsoft Research Cambridge}

\date{October 29, 2021}
\maketitle
% \makedeclaration

\begin{abstract} % 300 word limit

    The gene-expression history of an organism and its environment determine the organism's phenotype. The phenotype is an inherently qualitative state, deduced by relative biochemical concentration measurements collected by methods such as flow cytometry or fluorescence microscopy. The biochemical threshold concentrations that distinguish different phenotypes can be modelled by applying bifurcation analysis to differential equation models and the search for these boundaries in experimental data can be done using dimensionality reduction and clustering techniques. This establishes a relationship between bifurcations, phenotypes and machine learning techniques that are the subject of this thesis.

    % The first chapter presents an interactive tool for exploring phenotypes in flow cytometry data. In particular we explore a multi-tissue, high-dimensional, immune cell dataset. The tool bridges machine learning methods and the popular FlowJo, used to annotate cells with gating strategies. An assortment of dimensionality reduction techniques are applied to create two dimensional embeddings and confusion matrices are used to quantify annotation agreement between immunologists. By leveraging the geospatial mapping library OpenLayers to render, annotate and analyze cells, immunologists can now efficiently navigate the phenotype space of Human Cell Atlas datasets.
    
    % The next chapter focuses on a model-driven approach for exploring and designing phenotypes, where we demonstrate how model-guided design of synthetic E. Coli can elucidate pattern formation mechanisms in multicellular development. We infer the parameters of a biochemically motivated system of differential equations against time course fluorescence data acquired from plate reader experiments. Our design goals however were not in the temporal domain, rather we wanted to control the shape and size of a cusp bifurcation in the space of experimentally controlled input concentrations.
    
    % To address these limitations, I define a differentiable semi-supervised cost function that uses bifurcation locations as targets. Bifurcations are encouraged by an unsupervised term that extremises the curvature of the determinant of the Jacobian. By exploring the cost landscape for minimal models that span the space of saddle-nodes and pitchforks, I show that the parameter space basins define regions of qualitatively equivalent differential equations. The differentiability of the cost function enables efficient optimisation using libraries such as Flux.jl that leverage automatic differentiation. The impact of this work would enable experimentalists to efficiently navigate design spaces of differential equation models.
    
\end{abstract}
\newpage
\clearpage
\begin{center}
    \thispagestyle{empty}
    \vspace*{\fill}
    \epigraph{
        Blank pages are the worst.\\
        They impose a glaring responsibility onto someone to fill it with meaningful content.\\
        Much like other starting points:
        a new job, a marble stone, empty land, all future that tower over you, forcing the person facing it to ask:\\
        "do I really want to do this?"\\
        Blank pages are the worst.\\
        They reflect a glaring light upon you, reflecting the messages that would otherwise happinly bounce around in the ether, along with memories, desires, unfinished projects, and other concofonous bullshit in your head.\\
        A whole page!\\
        Rejoice at the etchings of achievement. You've started now so don't give up, or stop, you'll look stupid. Now make sure you go back and edit the previous page so that other will not know how stupid you are.\\
        Edit it, edit it,\\
        edit it into oblivion untill you can't recognise whether you are editing the page or yourself.}{}
    \vspace*{\fill}
\end{center}
\clearpage

\begin{acknowledgements}
    The past four years of my life have been full of excitement, creativity and exploration, none of which would have been possible without the support of Attila and Neil. Their mentorship and guidance has been a shining example of leadership and whose qualities I will take with me into future roles. I would like to thank my colleagues and friends at Microsoft Research, whose diverse projects in biotechnology captured my imagination.

    I would like to acknowledge Valerie Coppard, who was an absolute pleasure to work with and introduced me to Joanne Jones' lab at Cambridge University, catalysing the \emph{FlowAtlas.jl} project. Her enthusiasm...

    I would like to thank Matilda Peruzzo and Silvia Cabaliero During the pandemic. I would like to thank Mohammed Ali

    I am thankful to my friends at Burnt Umber who run a cozy and welcoming cafe in Hackney Wick, where I wrote a large chunk of my thesis, felt supported and cared for during difficult times. Disree Shaw my therapist. Fraser...

    I would like to thank family and dad
\end{acknowledgements}

%%%%%%%%%%%%%%%%%%%%%%%%%%%%%%%%%%%%%%%%
%%%%%%%%%%%%%%%%%%%%%%%%%%%%%%%%%%%%%%%%
%%%%%%%%%%%%%%%%%%%%%%%%%%%%%%%%%%%%%%%%
%%%%%%%%%%%%%%%%%%%%%%%%%%%%%%%%%%%%%%%%
%%%%%%%%%%%%%%%%%%%%%%%%%%%%%%%%%%%%%%%%

\setcounter{tocdepth}{2} 
% Setting this higher means you get contents entries for
%  more minor section headers.

\tableofcontents
\listoffigures