\title{Inferring bifurcations\\between phenotypes}
\author{Grisha Szep}

\department{Randall Division of Cell \& Molecular Biophysics}
\sponsor{Microsoft Research Cambridge}

\date{October 29, 2021}
\maketitle
% \makedeclaration

\begin{abstract} % 300 word limit

    The gene-expression history of an organism and its environment determine the organism’s phenotype. The phenotype is an inherently qualitative state, deduced by relative biochemical concentration measurements collected by methods such as flow cytometry or fluorescence microscopy. The biochemical threshold concentrations that distinguish different phenotypes can be modelled by applying bifurcation analysis to differential equation models and the search for these boundaries in experimental data can be done using dimensionality reduction and clustering techniques. This establishes a relationship between bifurcations, phenotypes and machine learning techniques that are the subject of this thesis.

    The first chapter presents an interactive tool for exploring phenotypes in flow cytometry data. In particular we explore a multi-tissue, high-dimensional, immune cell dataset. The tool bridges machine learning methods and the popular FlowJo, used to annotate cells with gating strategies. An assortment of dimensionality reduction techniques are applied to create two dimensional embeddings and confusion matrices are used to quantify annotation agreement between immunologists. By leveraging the geospatial mapping library OpenLayers to render, annotate and analyze cells, immunologists can now efficiently navigate the phenotype space of Human Cell Atlas datasets.
    
    The next chapter focuses on a model-driven approach for exploring and designing phenotypes, where we demonstrate how model-guided design of synthetic E. Coli can elucidate pattern formation mechanisms in multicellular development. We infer the parameters of a biochemically motivated system of differential equations against time course fluorescence data acquired from plate reader experiments. Our design goals however were not in the temporal domain, rather we wanted to control the shape and size of a cusp bifurcation in the space of experimentally controlled input concentrations.
    
    To address these limitations, I define a differentiable semi-supervised cost function that uses bifurcation locations as targets. Bifurcations are encouraged by an unsupervised term that extremises the curvature of the determinant of the Jacobian. By exploring the cost landscape for minimal models that span the space of saddle-nodes and pitchforks, I show that the parameter space basins define regions of qualitatively equivalent differential equations. The differentiability of the cost function enables efficient optimisation using libraries such as Flux.jl that leverage automatic differentiation. The impact of this work would enable experimentalists to efficiently navigate design spaces of differential equation models.
    
\end{abstract}
\newpage

\begin{acknowledgements}
    Acknowledgements to people I love
\end{acknowledgements}

%%%%%%%%%%%%%%%%%%%%%%%%%%%%%%%%%%%%%%%%
%%%%%%%%%%%%%%%%%%%%%%%%%%%%%%%%%%%%%%%%
%%%%%%%%%%%%%%%%%%%%%%%%%%%%%%%%%%%%%%%%
%%%%%%%%%%%%%%%%%%%%%%%%%%%%%%%%%%%%%%%%
%%%%%%%%%%%%%%%%%%%%%%%%%%%%%%%%%%%%%%%%

\setcounter{tocdepth}{2} 
% Setting this higher means you get contents entries for
%  more minor section headers.

\tableofcontents
\listoffigures
\listoftables